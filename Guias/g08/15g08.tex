\documentclass[11pt,spanish,a4paper]{article}
% Versión 1er cuat 2015 Víctor Bettachini < bettachini@df.uba.ar >

\usepackage{babel}
\addto\shorthandsspanish{\spanishdeactivate{~<>}}
\usepackage[utf8]{inputenc}
\usepackage{float}
\usepackage{units}
\usepackage{siunitx}
\usepackage{amsmath}
\usepackage{amstext}
\usepackage{amssymb}
\usepackage{graphicx}
\graphicspath{ {./graphs/} {../}}

\voffset-3.5cm
\hoffset-3cm
\setlength{\textwidth}{18cm}
\setlength{\textheight}{27cm}

\usepackage{lastpage}
\usepackage{fancyhdr}
\pagestyle{fancyplain}
\fancyhead{}
\fancyfoot{{\tiny \textcopyright Departamento de Física, FCEyN, UBA}}
\fancyfoot[C]{{\tiny Actualizado al \today}}
\fancyfoot[RO, LE]{Pág. \thepage/\pageref{LastPage}}
\renewcommand{\headrulewidth}{0pt}
\renewcommand{\footrulewidth}{0pt}


\begin{document}
\begin{center}
    \textsc{\large Física 2 (Físicos)} - Prof. Hernán Grecco - 1"er cuat. 2015\\
	\textsc{\large Guía 8:}	Ondas vectoriales: polarización
\end{center}

% Polarización: Antes de notación matricial una rotación "a mano". Un análisis "parados" en los ejes de polarización. (ej.3) Un ej. ¿Que pasa si rota tal elemento?

% \emph{LOS EJERCICIOS MARCADOS CON UN ASTERISCO (\textbf{*}) SON OPCIONALES}\\

\begin{enumerate}

\item Un par de polarizadores lineales están ubicados uno detrás de otro frente a una fuente de luz no polarizada.
	Sin recurrir al formalismo de matrices de rotación describa que sucede al rotar uno u otro.


\item Una onda inicialmente polarizada según \(x\) y viajando según \(z\) positivo incide en un polarizador cuyo eje de trasmisión forma un ángulo \(\alpha\) con el eje \(x\), y luego pasa por un segundo polarizador que forma un ángulo \(\beta\) con el primero.
¿Cuál es la expresión de la onda transmitida en los ejes originales?
¿Y en los ejes del segundo polarizador?
¿Cómo son las respectivas matrices que describen al sistema?
¿Cuál es la intensidad media transmitida por el sistema?


\item Se tiene un polarizador imperfecto con una matriz
% \[
\(
	\begin{bmatrix}
		t_x  & \epsilon \\
		\epsilon & \delta
	\end{bmatrix}
\)
% \]
con \(t_x \simeq 1\), \(\epsilon \ll 1 \) y \(\delta \ll 1 \).
\begin{enumerate}
	\item Hallar la matriz del polarizador si forma un ángulo \(\theta \) con el eje x.
	\item \label{ant} Hallar la energía transmitida si se incide con luz linealmente polarizada según x.
	\item Ídem. \ref{ant} si incide polarizada según y.
\end{enumerate}


\item Se tienen \(N\) polarizadores sucesivamente rotados en ángulos \(\pi/ 2N\), encontrar la matriz del sistema.
Hallar el límite para \(N\) tendiendo a \(\infty\).
Calcular la intensidad transmitida.


\item Una onda de polarización arbitraria incide sobre un espejo plano y se refleja sobre si misma.
¿Cómo escribe la onda reflejada cambiando al sistema \(z= -z\) de modo que nuevamente se propague según \(z\) positivo.
Note que al invertir \(z\), debe invertir algún otro eje para mantener las orientaciones relativas de los ejes.
¿Cómo es ahora la matriz de un polarizador en este nuevo sistema?
¿Y una lámina de onda?


\item Mostrar que el versor para una onda polarizada circularmente solo cambia en una fase ante una rotación de coordenadas.
¿Cuánto cambia la fase?
Explique.


\item Encontrar la matriz de un polarizador seguido de una lámina de cuarto de onda orientada con su eje a un ángulo \( \alpha \) respecto del eje del polarizador.
¿Cual es el estado de polarización de la onda transmitida?
¿Depende del estado de polarización de la incidente?
Explique.
Calcule la intensidad transmitida en función de \( \alpha \).


\item Al sistema anterior se le agrega un espejo que refleja la onda sobre si misma.
¿Para qué ángulo \( \alpha \) la transmisión del sistema a la vuelta es nula y para cuál es máxima?


\item Se realizan los siguientes experimentos: entre dos polarizadores rotados un ángulo \( \alpha \) se colocan alternativamente un medio con actividad óptica que rota la polarización un ángulo \( \beta \) y un rotador de Faraday que también rota ese mismo ángulo.
Calcule para cada caso la transmisión del sistema y para que valor de \( \beta \) es máxima.
Para ese valor de \( \beta \) se refleja la luz nuevamente sobre el sistema.
Calcule ahora la intensidad devuelta para cada caso.
¿En algún caso se puede hacer nula?


\item Se tiene un haz de luz y se quiere conocer su estado de polarización (el tipo de polarización y la orientación respecto de los ejes del laboratorio) realizando experimentos.
Se cuenta con el siguiente material: un detector que mide intensidad de luz, un polarizador lineal con el eje de transmisión paralelo a la mesa óptica (la mesa sobre la que se trabaja), una lámina de media onda, y una lámina de cuarto de onda.
Las dos últimas están montadas en soportes que permiten girarlas, y se conoce la ubicación de los ejes ordinarios.
Describa un procedimiento experimental que contemple todos los casos que puedan presentarse.


\item  Incide un haz de luz linealmente polarizada sobre la superficie de separación de dos medios transparentes.
¿Qué condiciones deben cumplirse para que ese haz se transmita totalmente hacia el segundo medio?


\item  Un haz de luz circularmente polarizada en sentido horario incide con el ángulo de polarización sobre la superficie de separación de dos medios transparentes.
¿Cuál es el estado de polarización del haz reflejado?
¿Y del transmitido?
Justifique.
	   

\item Sobre una superficie de separación entre dos medios de índices \(n_1 \) y \(n_2 \) (con \(n_1 > n_2 \) ), incide un rayo desde el medio \(n_1 \).
	\begin{enumerate}
		\item ¿Cuál es el ángulo de incidencia crítico a partir del cual se produce reflexión total?
		\item ¿Cuál es el ángulo de polarización?
		\item ¿Es posible que el ángulo de polarización sea mayor que el ángulo crítico?
			Justifique físicamente y analíticamente.
	\end{enumerate}



\end{enumerate}
\end{document}
