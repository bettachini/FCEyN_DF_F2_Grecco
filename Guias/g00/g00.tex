\documentclass[11pt,spanish,a4paper]{article}
% Versión 1er cuat 2014 Víctor Bettachini < bettachini@df.uba.ar >

\usepackage{babel}
\addto\shorthandsspanish{\spanishdeactivate{~<>}}
\usepackage[utf8]{inputenc}
\usepackage{amsmath}
\usepackage{amstext}
\usepackage{amssymb}
\usepackage{graphicx}

\voffset-3.5cm
\hoffset-3cm
\setlength{\textwidth}{17.5cm}
\setlength{\textheight}{27cm}

\usepackage{lastpage}
\usepackage{fancyhdr}
\pagestyle{fancyplain}
\fancyhead{}
\fancyfoot{}
\fancyfoot[C]{ {\tiny Actualizado al \today} }
\fancyfoot[RO, LE]{Pág. \thepage/\pageref{LastPage}}
\renewcommand{\headrulewidth}{0pt}
\renewcommand{\footrulewidth}{0pt}


\begin{document}
\begin{center}
\textsc{\large Física 2 (Físicos)} - Prof. Hernán Grecco\\
\textsc{\large Primer Cuatrimestre - 2014}\\
\textsc{\large Guía 0:} Repaso de matemática
\end{center}

\begin{enumerate}

\item Desarrollar a 2"o orden:
\begin{enumerate}
	\item \( \sqrt{a^{2}+x^{2}} \) alrededor de \( x=0 \),  \( x\ll a \)
	\item \( \left(a^{2}+x^{2}\right)^{-\frac{1}{2}} \) alrededor de \(x=0 \), \(x \ll a \)
	\item \( \sen(kx) \) alrededor de \(x=0 \), \(kx \ll 1 \)
	\item \( \sen\left[k(x+d)\right] \) a orden 0, alrededor de \(x= x_{0} \) ¿Qué condición debe pedir?
	\item \( \mathrm{e}^{kx} \) alrededor de \(x=0 \), \(kx \ll 1 \)
	\item \( \left(a+x\right)^{-1} \) alrededor de \(x=0 \), \(x\ll a \)
\end{enumerate}

\item Integrar
\begin{enumerate}
	\item \(\int_{a}^{b} \mathrm{e}^{cx+d}\, \mathrm{d}x \)
	\item \(\int_{a}^{b} \cos\left(kx+\varphi\right)\, \mathrm{d}x\)
	\item \(\int_{a}^{b} x \cos\left(kx+\varphi\right)\, \mathrm{d}x \)
	\item \(\int_{a}^{b} \mathrm{e}^{cx+d} \cos\left(kx+\varphi\right)\, \mathrm{d}x\)
	\item \(\int_{a}^{b} \mathrm{e}^{cx+d} \left(\alpha+\beta x+\gamma x^{2}\right)\, \mathrm{d}x\)
\end{enumerate}

\item Graficar esquemáticamente y hallar los ceros
\begin{enumerate}
	\item \(\mathrm{e}^{cx+d} \cos\left(kx+ \varphi\right) \)
	\item \(\mathrm{e}^{cx+d} \sen\left(kx+ \varphi\right) \)
\end{enumerate}

\item Probar que, dadas las constantes reales \(A_{1} \), \(A_{2} \), \(\varphi_{1} \) y \(\varphi_{2} \), existen constantes \(A \) y \( \varphi \) tal que se cumple la siguiente igualdad:
\[
A_{1}\cos\left(kx+\varphi_{1}\right)+A_{2}\cos\left(kx+\varphi_{2}\right)=A\cos\left(kx+\varphi\right)
\]

\item Discutir si es posible satisfacer la siguiente igualdad. En caso de que lo sea, hallar \(A \), \(\omega \) y \(\varphi \) en función de \(A_{1} \), \(A_{2} \), \(\varphi_{1} \),  \(\varphi_{2} \), \(\omega_1 \) y \(\omega_2 \)
\[
A_{1}\cos\left(\omega_1 t+\varphi_{1}\right)+A_{2}\cos\left(\omega_2 t+ \varphi_{2}\right)= A\cos\left(\omega t+ \varphi\right)
\]

\item Discutir, en función del parámetro (\(\lambda \)), el siguiente sistema:
\begin{align*}
x+ 2y+ \lambda z & =-3\\
3x- 2y- 4z & = -\lambda\\
-7x+ 2y+ 4z & = -2
\end{align*}
Resolver cuando sea posible.

\item Encuentre para \(z \) su parte real (\( \Re\mathfrak{e}\, z\)), módulo (\( \left| z \right| \)), fase (\( \theta\) en \(z=\mathrm{e}^{i \theta} \)) y su conjugado (\( \bar{z} \))
\begin{enumerate}
	\item \(z=(a+ i b)^{-1} \)
	\item \(z=\rho \mathrm{e}^{i \phi} \mathrm{e}^{i \omega t} \)
	\item \(z= \mathrm{e}^{a+ i b} \)
	\item \(z= \mathrm{e}^{i \varphi}+ \mathrm{e}^{i \phi} \)
	\item \(z= A \mathrm{e}^{i \varphi}+ B \mathrm{e}^{i \phi} \)
\end{enumerate}
siendo \(A \), \(B \), \(\rho \), \(\varphi \) y \(\phi \) reales.

\end{enumerate}

\end{document}
