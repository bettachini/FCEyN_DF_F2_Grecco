\documentclass[11pt,spanish,a4paper]{article}
% Versión 1er cuat 2014 Víctor Bettachini < bettachini@df.uba.ar >

\usepackage{babel}
\addto\shorthandsspanish{\spanishdeactivate{~<>}}
\usepackage[utf8]{inputenc}
\usepackage{float}
\usepackage{units}
\usepackage{siunitx}
\usepackage{amsmath}
\usepackage{amstext}
\usepackage{amssymb}
\usepackage{graphicx}
\graphicspath{ {./graphs/} {../}}

\voffset-3.5cm
\hoffset-3cm
\setlength{\textwidth}{17.5cm}
\setlength{\textheight}{27cm}

\usepackage{lastpage}
\usepackage{fancyhdr}
\pagestyle{fancyplain}
\fancyhead{}
\fancyfoot{}
\fancyfoot[C]{ {\tiny Actualizado al \today} }
\fancyfoot[RO, LE]{Pág. \thepage/\pageref{LastPage}}
\renewcommand{\headrulewidth}{0pt}
\renewcommand{\footrulewidth}{0pt}


\begin{document}
\begin{center}
	\textsc{\large Física 2 (Físicos)} - Prof. Hernán Grecco\\
	\textsc{\large Primer Cuatrimestre - 2014}\\
	\textsc{\large Guía 11:} Difracción por objectos periódicos
\end{center}

% \emph{LOS EJERCICIOS MARCADOS CON UN ASTERISCO (\textbf{*}) SON OPCIONALES}\\

% \textbf{Difracción por objectos periódicos}

\begin{enumerate}	

	\item En una rendija de ancho \( D\) se ubican sucesivamente distintas diapositivas de transmisión \( t(x,y) \).
		Calcular para cada caso el perfil de intensidades en un plano ubicado suficientemente lejos como para que valga la aproximación de Fraunhofer.
		Discutir cualitativamente los resultados.
		Proponga para cada caso como construirlas.
	\begin{enumerate}	
		\item \( t(x,y)= \cos{ (\alpha x) } \)
		\item \( t(x,y)= 2 \cos{ (\alpha x) } \)
		\item \( t(x,y)= \mathrm{e}^{i \alpha x} \)
		\item \( t(x,y)= \mathrm{e}^{-x^2/ d^2 } \) con \( d \ll D \)
	\end{enumerate}	

		
\item \label{old_2} Con una onda plana se ilumina en forma normal una diapositiva de estructura periódica.
	Si se iluminan \( N \) períodos, calcular en la aproximación de Fraunhofer la amplitud y la intensidad en una pantalla ubicada a una distancia \( L \) de la diapositiva, para cada una de las siguientes transmisiones de las mismas:
	\begin{enumerate}	
		\item \( t(x)= \cos{ (K_0 x) } \)
		\item \( t(x)= 1+ \cos{ (K_0 x) } \)
		\item \( t(x)= 1+ \sen{ (K_0 x) } \)
		\item \( t(x)= 1+ \cos{ (K_0 x) }+ \sen{ (K_0 x) } \)
		\item \( t(x)= 1+ \cos{ (K_0 x) }+ \sen{ (2 K_0 x) } \)
	\end{enumerate}	
	Discutir las similitudes, diferencias y algún sistema sencillo para generarlas.
	
	
\item \label{old_3} \( N \) ranuras de ancho \( a \) y separación \( b \) son iluminadas uniformemente.
	Calcular la	figura de difracción en el campo lejano.
	Discutir como cambia el patrón de intensidades si se cambia el ángulo con que se incide sobre la red.
	¿Qué pasa si inciden dos longitudes de onda distintas, y en qué casos se distinguen los dos máximos?
	
	
\item Para los dos ejercicios anteriores de ejemplos alternativos de redes que den el mismo	patrón de intensidad.
	
	
\item \label{old_5} ¿Como cambian las figuras de difracción si en los ejercicios \ref{old_2} y \ref{old_3} se ilumina con un haz Gaussiano?
	¿Qué es propio de la forma de iluminar, qué de la periodicidad de la transparencia y qué de la forma particular que se repite periódicamente?
	
	
\item Repita los ejercicios anteriores (\ref{old_2}-\ref{old_5}) para el caso en que se intercala una lente después de la red.
	¿Y si se intercala antes?
	

\item Un espejo tiene una superficie ondulada de modo que la fase de la onda reflejada varía según \( \varphi= \delta \cos{ (K_0 x) } \).
	Si \( \delta \ll 1 \), calcule la onda difractada en la reflexión.

	
\item Se tiene una estructura periódica con una transmisión
	\[
		t(x,y)= \left[ 1 + \cos{ (K_1 x) } \right] \left[ 1 + \cos{ (K_2 y ) } \right],
	\]
	calcular el perfil de intensidades difractado en la aproximación de campo lejano.

	
\item Calcular la dirección en que aparecen los máximos en una estructura bidimensional en que \( \vec{a} = \frac{\lambda}{3} \hat{x} \) y \( \vec{b} = \frac{\lambda}{5} \hat{y} \).


\end{enumerate}	

\end{document}
