\documentclass[11pt,spanish,a4paper]{article}
% Versión 1er cuat 2015 Víctor Bettachini < bettachini@df.uba.ar >

\usepackage{babel}
\addto\shorthandsspanish{\spanishdeactivate{~<>}}
\usepackage[utf8]{inputenc}
\usepackage{float}
\usepackage{units}
\usepackage{siunitx}
\usepackage{amsmath}
\usepackage{amstext}
\usepackage{amssymb}
\usepackage{graphicx}
\graphicspath{ {./graphs/} {../}}

\voffset-3.5cm
\hoffset-3cm
\setlength{\textwidth}{17.5cm}
\setlength{\textheight}{27cm}

\usepackage{lastpage}
\usepackage{fancyhdr}
\pagestyle{fancyplain}
\fancyhead{}
\fancyfoot{{\tiny \textcopyright Departamento de Física, FCEyN, UBA}}
\fancyfoot[C]{{\tiny Actualizado al \today}}
\fancyfoot[RO, LE]{Pág. \thepage/\pageref{LastPage}}
\renewcommand{\headrulewidth}{0pt}
\renewcommand{\footrulewidth}{0pt}


\begin{document}
\begin{center}
    \textsc{\large Física 2 (Físicos)} - Prof. Hernán Grecco - 1"er cuat. 2015\\
	\textsc{\large Guía 10:} Difracción por un sistema simple
\end{center}

% \emph{LOS EJERCICIOS MARCADOS CON UN ASTERISCO (\textbf{*}) SON OPCIONALES}\\

% \textbf{Difracción por objectos sencillos}


\begin{enumerate}

\item Una ranura de ancho \( D \) es iluminada por una onda plana incidente perpendicularmente al plano de la ranura, observándose la figura de difracción en una pantalla ubicada a una distancia \( L \).
	\begin{enumerate}
		\item ¿A qué distancia debe ubicarse la pantalla para que valga la aproximación de Fraunhofer?
		\item Estime dichas distancias para ranuras de ancho \SI{10}{\( \mu m \)}, \SI{100}{\( \mu m \)} y \SI{1}{mm}, para luz visible.
		\item ¿Para qué ranuras y distancias se cumplen condiciones equivalentes con ondas de sonido?
		\item \label{ant_d} Para alguno de los casos anteriores calcule y grafique como cambia la distribución de intensidades si la onda incide con un ángulo de \SI{10}{\(^\circ\)} respecto de la normal al plano de la ranura.
		\item Ídem. \ref{ant_d}, si la ranura se ilumina con una fuente puntual ubicada a una distancia \( L' \).
	\end{enumerate}
	

\item Delante de la ranura del problema anterior se ubica una lente de distancia focal \( f \).
	\begin{enumerate}
		\item Calcule el perfil de intensidad en una pantalla plana ubicada en el plano focal de la	lente.
		\item Lo mismo si la onda incide con un ángulo de \SI{10}{\(^\circ\)}.
		\item Si inciden ambas ondas, ¿qué condiciones debe cumplir la lente para que las manchas respectivas queden nítidamente separadas?
		\item ¿Con qué precisión debe ubicarse la pantalla en el plano focal para que valgan los resultados de los puntos anteriores?
	\end{enumerate}


\item Una fuente puntual está ubicada a una distancia s de una lente de distancia focal \( f \).
	\begin{enumerate}	
		\item Calcule la función de onda en el plano imagen.
		\item Calcule el perfil de intensidad.
			¿Qué información se perdió al medir la intensidad?
	\end{enumerate}	

	
\item Se tiene una onda monocromática de perfil Gaussiano que en \( z= 0 \) tiene la forma
	\[
		\Psi( x, y, t ) = A \mathrm{e}^{ i \omega t } \mathrm{e}^{- ( x + y ) / 2 \sigma }.
	\]
	\begin{enumerate}	
		\item Calcular en la aproximación paraxial (integral de Kirchhoff) la función de onda en un plano \( z=\mathrm{cte.} \) cualquiera.
		\item Calcular el perfil de intensidad en dicho plano.
			¿Qué información se pierde al medir la intensidad?
		\item Con el dato de la función de onda en el plano \( z \), calcule nuevamente la función de onda en un plano \( z’ \) posterior.
			Notar como se recupera el ya calculado originalmente para todo \( z \).
			¿Porqué no puedo calcularlo si conozco solamente el perfil de intensidades?
	\end{enumerate}	

		
\item En el plano \( z \) del problema anterior se ubica una lente.
	\begin{enumerate}	
		\item Calcular la función de onda ahora en un plano a una distancia \( z’ \) de la lente.
			Escriba la expresión integral y resuelva suponiendo la lente de diámetro mucho mayor que el haz Gaussiano.
			Discuta como cambia según la ubicación de la lente.
			¿Como es el perfil de intensidades en el foco de la lente?
			¿Para qué valor de \( z \) dicho perfil es mas angosto?
		\item Para el problema anterior, encuentre la nueva cintura del haz (el plano de mínimo ancho espacial).
		\item Discuta el caso particular en que \( z= f \) (lente ubicada con el foco en la cintura del haz).
		\item ¿Como cambia el punto a si la lente tiene diámetro mucho menor que el diámetro característico del haz.
	\end{enumerate}	

\end{enumerate}	

\end{document}
